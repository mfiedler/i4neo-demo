% Set to Handout if filename contains "_handout"
\RequirePackage{substr}
\IfSubStringInString{\detokenize{_handout}}{\jobname}{\PassOptionsToClass{handout}{beamer}}{}

\documentclass[10pt]{beamer}
% or try: \documentclass[aspectratio=1610,12pt]{beamer}
\usetheme[book]{neo}

\usepackage{appendixnumberbeamer}

\usepackage{booktabs} % nice tables
\usepackage[scale=2]{ccicons} % creative commons icons

\usepackage{pgfplots} % draw all kinds of fancy plots
\usepgfplotslibrary{dateplot}

\usepackage{xspace} % space macro for the following command
\newcommand{\themename}{\textbf{\textsc{neo}}\xspace}

\usepackage{listings} % code listings

% Multiple slides on a page if filename contains "_1x2" or "_2x2"
\IfSubStringInString{\detokenize{_2x2}}{\jobname}{
	\pgfpagesuselayout{4 on 1}[letterpaper,landscape,border shrink=2.5mm]
	\neoset{background=white}
}{
	\IfSubStringInString{\detokenize{_1x2}}{\jobname}{
		\pgfpagesuselayout{2 on 1}[letterpaper,border shrink=5mm]
		\neoset{background=white}
	}{}
}

% Show notes in neo
\neoset{notes=show}

\title[i4\themename Theme Demo]{i4\themename}
\subtitle{Proposal for a modern Beamer theme}
\date{\today}
\author{Wolfgang Händler}
\institute{Friedrich-Alexander-Universität Erlangen-Nürnberg}
\titlegraphic{\includegraphics[height=1cm]{theme/images/logo-i4.pdf} \hfill \includegraphics[height=0.8cm]{theme/images/logo-fau-tf.pdf}}
%\neoset{progressbar=foot}

\begin{document}

\maketitle[intern]

\begin{frame}{Table of Contents}
  \setbeamertemplate{section in toc}[sections numbered]
  \tableofcontents[hideallsubsections]
\end{frame}

\section{Introduction}

\begin{frame}[fragile]
  The \themename theme is a Beamer theme with minimal visual noise
  based on the \href{https://github.com/matze/mtheme}{\textsc{Metropolis} Beamer Theme} by Matthias Vogelgesang and inspired by the \href{https://github.com/hsrmbeamertheme/hsrmbeamertheme}{\textsc{hsrm} Beamer Theme} by Benjamin Weiss.

  Enable the theme by loading

  \begin{verbatim}    \documentclass{beamer}
    \usetheme{neo}\end{verbatim}

\end{frame}
\begin{frame}[fragile]{Sections}
  Sections group slides of the same topic

  \begin{verbatim}    \section{Elements}\end{verbatim}

  for which \themename provides a nice progress indicator \ldots
\end{frame}

\begin{frame}[fragile]{Presentation notes}
  The theme has built-in support for notes

  \begin{verbatim}    \pnote{Important presentation note}\end{verbatim}

  \themename can not only create embedded \href{https://github.com/pdfpc/pdfpc}{\textsc{pdfpc}} notes with
  \begin{verbatim}    \neoset{notes=pdfpc}\end{verbatim}
  but also full note pages (e.g., for printing)
  \begin{verbatim}    \neoset{notes=show}\end{verbatim}
  \pnote{Hi there!}
  \pnote{We have some text in \textbf{here}!}
  \pnote{Try to use \texttt{notes=preview-right} with \texttt{pdfpc --notes=right ./out/presentation.pdf}}
  \pnote<1>{By the way, the command is overlay-aware!}
\end{frame}

\section{Elements}

\begin{frame}[fragile]{Typography}
      \begin{verbatim}The theme provides sensible defaults to
\emph{emphasize} text, \alert{accent} parts
or show \textbf{bold} results.\end{verbatim}

  \begin{center}becomes\end{center}

  The theme provides sensible defaults to \emph{emphasize} text,
  \alert{accent} parts or show \textbf{bold} results.
\end{frame}

\begin{frame}{Font feature test}
  \begin{itemize}
    \item Regular
    \item \textit{Italic}
    \item \textsc{SmallCaps}
    \item \textbf{Bold}
    \item \textbf{\textit{Bold Italic}}
    \item \textbf{\textsc{Bold SmallCaps}}
    \item \texttt{Monospace}
    \item \texttt{\textit{Monospace Italic}}
    \item \texttt{\textbf{Monospace Bold}}
    \item \texttt{\textbf{\textit{Monospace Bold Italic}}}
  \end{itemize}
\end{frame}

\newcommand{\colorsample}[1]{\textcolor{nDark#1}{nDark#1} & \textcolor{n#1}{n#1} & \textcolor{nLight#1}{nLight#1}}
\begin{frame}{Predefined Colors}
  \begin{center}
    \begin{tabular}{ccc}
      \colorsample{Red} \\
      \colorsample{Green} \\
      \colorsample{Blue} \\
      \colorsample{Cyan} \\
      \colorsample{Yellow} \\
      \colorsample{Grey} \\
      \textcolor{nBlack}{nBlack} \\
      \colorbox{nBlack}{\textcolor{nWhite}{nWhite}} \\
    \end{tabular}
  \end{center}
\end{frame}

\begin{frame}{Lists}
  \begin{columns}[T,onlytextwidth]
    \column{0.33\textwidth}
      Items
      \begin{itemize}
        \item Milk \item Eggs \item Potatos
      \end{itemize}

    \column{0.33\textwidth}
      Enumerations
      \begin{enumerate}
        \item First, \item Second and \item Last.
      \end{enumerate}

    \column{0.33\textwidth}
      Descriptions
      \begin{description}
        \item[PowerPoint] Meeh. \item[Beamer] Yeeeha.
      \end{description}
  \end{columns}
\end{frame}
\begin{frame}{Animation}
  \begin{itemize}[<+->]
    \item \alert<4>{This is\only<4>{ really} important}
    \item Now this
    \item And now this
  \end{itemize}
\end{frame}
\begin{frame}{Figures}
  \begin{figure}
    \newcounter{density}
    \setcounter{density}{20}
    \begin{tikzpicture}
      \def\couleur{alerted text.fg}
      \path[coordinate] (0,0)  coordinate(A)
                  ++( 90:5cm) coordinate(B)
                  ++(0:5cm) coordinate(C)
                  ++(-90:5cm) coordinate(D);
      \draw[fill=\couleur!\thedensity] (A) -- (B) -- (C) --(D) -- cycle;
      \foreach \x in {1,...,40}{%
          \pgfmathsetcounter{density}{\thedensity+20}
          \setcounter{density}{\thedensity}
          \path[coordinate] coordinate(X) at (A){};
          \path[coordinate] (A) -- (B) coordinate[pos=.10](A)
                              -- (C) coordinate[pos=.10](B)
                              -- (D) coordinate[pos=.10](C)
                              -- (X) coordinate[pos=.10](D);
          \draw[fill=\couleur!\thedensity] (A)--(B)--(C)-- (D) -- cycle;
      }
    \end{tikzpicture}
    \caption{Rotated square from
    \href{http://www.texample.net/tikz/examples/rotated-polygons/}{texample.net}.}
  \end{figure}
\end{frame}
\begin{frame}{Tables}
  \begin{table}
    \caption{Largest cities in the world (source: Wikipedia)}
    \begin{tabular}{@{} lr @{}}
      \toprule
      City & Population\\
      \midrule
      Mexico City & 20,116,842\\
      Shanghai & 19,210,000\\
      Peking & 15,796,450\\
      Istanbul & 14,160,467\\
      \bottomrule
    \end{tabular}
  \end{table}
\end{frame}
\begin{frame}{Blocks}
  Three different block environments are pre-defined and may be styled with an
  optional background color.

  \begin{columns}[T,onlytextwidth]
    \column{0.5\textwidth}
      \begin{block}{Default}
        Block content.
      \end{block}

      \begin{alertblock}{Alert}
        Block content.
      \end{alertblock}

      \begin{exampleblock}{Example}
        Block content.
      \end{exampleblock}

    \column{0.5\textwidth}

      \neoset{block=fill}

      \begin{block}{Default}
        Block content.
      \end{block}

      \begin{alertblock}{Alert}
        Block content.
      \end{alertblock}

      \begin{exampleblock}{Example}
        Block content.
      \end{exampleblock}

  \end{columns}
\end{frame}
\begin{frame}[fragile]{Code}
  \begin{lstlisting}[language=C,basicstyle=\ttfamily]
#include <stdio.h>

int main() {
    printf("Hello World!");
    return 0;
}
  \end{lstlisting}
\end{frame}
\begin{frame}{Math}
  \begin{equation*}
    e = \lim_{n\to \infty} \left(1 + \frac{1}{n}\right)^n
  \end{equation*}
\end{frame}
\begin{frame}{Line plots}
  \begin{figure}
    \begin{tikzpicture}
      \begin{axis}[
        mlineplot,
        width=0.9\textwidth,
        height=6cm,
      ]

        \addplot {sin(deg(x))};
        \addplot+[samples=100] {sin(deg(2*x))};

      \end{axis}
    \end{tikzpicture}
  \end{figure}
\end{frame}
\begin{frame}{Bar charts}
  \begin{figure}
    \begin{tikzpicture}
      \begin{axis}[
        mbarplot,
        xlabel={Foo},
        ylabel={Bar},
        width=0.9\textwidth,
        height=6cm,
      ]

      \addplot plot coordinates {(1, 20) (2, 25) (3, 22.4) (4, 12.4)};
      \addplot plot coordinates {(1, 18) (2, 24) (3, 23.5) (4, 13.2)};
      \addplot plot coordinates {(1, 10) (2, 19) (3, 25) (4, 15.2)};

      \legend{lorem, ipsum, dolor}

      \end{axis}
    \end{tikzpicture}
  \end{figure}
\end{frame}
\begin{frame}{Quotes}
  \begin{quote}
    Veni, Vidi, Vici
  \end{quote}
\end{frame}

{%

\neoset{frametitle icon=i4}
\neoset{footer style=standout}
\neoset{footer=author title}
\begin{frame}[fragile]{Header icon and frame footer}
    You can embed the i4 (or FAU) logo on the top right corner using
    \verb|    \neoset{frametitle icon=i4}|

    A footer for conferences can be displayed using
    \verb|    \neoset{footer=author title}|

    In adddition, \themename defines a custom beamer template to add a text to the footer. It can be set via
    \verb|    \setbeamertemplate{frame footer}{My custom footer}|

    To change the background color of the footer, simply use \verb|    \neoset{footer style=standout}|
\end{frame}

\begin{frame}[fragile]{Header icon}{with a frame subtitle}
    Be cautious when mixing framtitel icons and frame subtitles.

    If a frame subtitle is given for the current slide, the frametitle icon is replaced.
\end{frame}
}

\begin{frame}{References}
  Some references to showcase [allowframebreaks] \cite{knuth92,ConcreteMath,Simpson,Er01,greenwade93}
  References can also be added in place as footnote\footnote{Just use a normal footnote which contains lots and lots of text and can be created using \texttt{$\backslash$footnote\{\}}}
\end{frame}

% Not visible in handout
\begin{frame}<handout:0>[standout]
  Standout!

  Next comes the conclusion!
\end{frame}

\section{Conclusion}

\begin{frame}{Summary}

  Get the source of this theme from

  \begin{center}\url{https://gitlab.cs.fau.de/i4/tex/i4neo.git}\end{center}

  The theme \emph{itself} is licensed under a
  \href{http://creativecommons.org/licenses/by-sa/4.0/}{Creative Commons
  Attribution-ShareAlike 4.0 International License}.

  \begin{center}\ccbysa\end{center}

  \visible<2>{
    \neoset{block=fill}
    \begin{center}
      \begin{block}{Questions?}
        user@cs.fau.de
      \end{block}
    \end{center}
  }

\end{frame}

\appendix

\begin{frame}[fragile]{Backup slides}
  Sometimes, it is useful to add slides at the end of your presentation to
  refer to during audience questions.

  The best way to do this is to include the \verb|appendixnumberbeamer|
  package in your preamble and call \verb|\appendix| before your backup slides.

  \themename will automatically turn off slide numbering and progress bars for
  slides in the appendix.
\end{frame}

\begin{frame}[allowframebreaks]{References}

  \bibliography{demo}
  \bibliographystyle{abbrv}

\end{frame}

\end{document}
